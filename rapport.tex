\documentclass[12pt, a4paper]{report}

\usepackage[utf8]{inputenc}
\usepackage[T1]{fontenc}
\usepackage[english]{babel}
\usepackage{geometry}
\geometry{a4paper, margin=1in, headheight=14.5pt} % Adjusted margins with proper headheight
\usepackage{pdfpages}
\usepackage{emptypage} % To ensure blank pages are truly empty
\usepackage{microtype} % Improve justification and reduce overfull hboxes
\usepackage{titlesec} % For custom chapter and section styling
\usepackage{xcolor} % For custom colors
\usepackage[most]{tcolorbox} % For colored boxes
\usepackage{fancyhdr} % For custom headers and footers
\usepackage{tocloft} % For Table of Contents customization
\usepackage{lettrine} % For drop caps
\usepackage{lipsum} % For dummy text if needed
\usepackage{enumitem} % For better list control
\usepackage{caption} % For caption styling
\usepackage{setspace} % For line spacing control
\usepackage{pifont} % For decorative symbols
\usepackage{tikz} % For drawing fancy elements
\usepackage{amssymb} % For mathematical symbols like \checkmark
\usepackage{bookmark} % Better PDF bookmarks
\usepackage{hyperref} % For clickable links in ToC (moved to after other packages)
\usepackage{float}
\usepackage{colortbl} % For colored table cells
\usepackage{array} % For better table column formatting
\usepackage{calc} % For coordinate calculations in TikZ
\usepackage{mdframed} % For framed environments
\usepackage{afterpage} % For operations on the following page
\usepackage{longtable} % For tables that span multiple pages
\usetikzlibrary{calc}
\tcbuselibrary{skins,breakable}

% Define a refined professional color palette
\definecolor{primary}{RGB}{33, 64, 95} % Deep navy blue
\definecolor{secondary}{RGB}{145, 145, 145} % Medium gray
\definecolor{accent}{RGB}{201, 172, 140} % Gold/Bronze accent
\definecolor{background}{RGB}{248, 248, 248} % Light background
\definecolor{bodytext}{RGB}{40, 40, 40} % Near-black for body text

% Set default text color
\color{bodytext}

% Customize chapter style - more elegant
\titleformat{\chapter}[display]
{\normalfont\Large\bfseries\color{primary}}
{\filcenter\color{primary}\MakeUppercase{\chaptertitlename}~\thechapter}
{1em}
{\filcenter\LARGE}[\vspace{0.5em}\centerline{\rule{0.6\textwidth}{1pt}}]
\titlespacing*{\chapter}{0pt}{30pt}{40pt}

% Section styling - more subtle
\titleformat{\section}
{\normalfont\large\bfseries\color{primary}}
{\thesection}{1em}{}
\titlespacing*{\section}{0pt}{3.5ex plus 1ex minus .2ex}{2.3ex plus .2ex}

% Subsection styling
\titleformat{\subsection}
{\normalfont\normalsize\bfseries\color{primary}}
{\thesubsection}{1em}{}

% Set up fancy headers - more elegant
\pagestyle{fancy}
\fancyhf{}
\fancyhead[R]{\thepage}
\fancyhead[L]{\textit{\leftmark}}
\fancyfoot[C]{\textcolor{secondary}{\small{Final Year Project}}}
\renewcommand{\headrulewidth}{0.4pt}
\renewcommand{\footrulewidth}{0.2pt}

% Enhanced Table of Contents styling
% Title formatting
\renewcommand{\cfttoctitlefont}{\hfill\LARGE\bfseries\color{primary}} 
\renewcommand{\cftaftertoctitle}{\hfill\null\\[1em]\hfill\rule{0.6\textwidth}{1pt}}

% Entry formatting
\renewcommand{\cftchapfont}{\normalfont\bfseries\color{bodytext}} % Chapter font in black
\renewcommand{\cftsecfont}{\normalfont\itshape} % Section font
\renewcommand{\cftsubsecfont}{\normalfont} % Subsection font

% Dot leaders with custom styling
\renewcommand{\cftdot}{$\cdot$} % Better centered dots
\renewcommand{\cftdotsep}{2.0} % Tighter dot spacing
\renewcommand{\cftchapleader}{\color{secondary}\cftdotfill{\cftdotsep}} % Chapter dots
\renewcommand{\cftsecleader}{\color{secondary}\cftdotfill{\cftdotsep}} % Section dots

% Page numbers
\renewcommand{\cftchappagefont}{\normalfont\bfseries\color{bodytext}} % Chapter page number
\renewcommand{\cftsecpagefont}{\normalfont\color{bodytext}} % Section page number

% Spacing for better presentation
\setlength{\cftbeforetoctitleskip}{1em} % Space before title
\setlength{\cftaftertoctitleskip}{2em} % Space after title
\setlength{\cftbeforechapskip}{1em} % Space before chapters
\setlength{\cftbeforesecskip}{0.5em} % Space before sections

% Chapter numbering format
\renewcommand{\cftchappresnum}{\color{primary}} % Color for chapter numbers
\renewcommand{\cftchapaftersnum}{.\hspace{0.8em}} % Add period after number

% Indentation
\setlength{\cftchapindent}{0em} % No indent for chapters
\setlength{\cftsecindent}{1.5em} % Indent for sections
\setlength{\cftsubsecindent}{3em} % Indent for subsections

% Title
\renewcommand{\contentsname}{\color{primary}Table of Contents}

% Improved caption style
\captionsetup{
  font=small,
  labelfont={bf,color=primary},
  margin=10pt,
  format=hang,
  justification=centering,
}

% Lettrine (drop caps) settings - more elegant
\renewcommand{\LettrineFontHook}{\color{primary}\bfseries}
\setcounter{DefaultLines}{3}
\setlength{\DefaultFindent}{0.5em}
\setlength{\DefaultNindent}{0em}
\setlength{\DefaultSlope}{0pt} % Ensure proper alignment

% Define line spacing for main text
\onehalfspacing

% Hyperref settings - more professional
\hypersetup{
    colorlinks=true,
    linkcolor=primary,
    filecolor=primary,
    urlcolor=primary,
    citecolor=primary,
    pdftitle={PFE Rapport},
    pdfauthor={KORPOR},
    pdfsubject={Final Year Project},
    pdfkeywords={MySQL, Express-Node.js, React, Vite},
    pdfpagemode=UseOutlines,
    pdfstartview=FitH,
    bookmarksopen=true,
    bookmarksnumbered=true,
}

% Define a new environment for important information - more elegant
\newenvironment{important}{%
  \begin{tcolorbox}[
    colback=background,
    colframe=primary,
    arc=1mm, % Slightly rounded corners
    boxrule=0.5pt,
    left=8pt,
    right=8pt,
    top=6pt,
    bottom=6pt,
    width=\textwidth,
    title={\textbf{Important}},
    fonttitle=\color{white}
  ]
}{%
  \end{tcolorbox}
}

% Define a command for first paragraph styling with better alignment
\newcommand{\firstparagraph}[1]{%
  \lettrine[lines=2,findent=0.5em,nindent=0em,loversize=0.05,lraise=0.05]{\textcolor{primary}{#1}}{}%
}

% Define a command for chapter quotes
\newcommand{\chapterquote}[2]{%
  \begin{flushright}
    \begin{minipage}{0.7\textwidth}
      \small\itshape #1
      \begin{flushright}
        \normalfont --- #2
      \end{flushright}
    \end{minipage}
  \end{flushright}
  \vspace{1cm}
}

% Enhanced figure environment with side lines
\newtcolorbox{figureframe}{
  enhanced,
  frame hidden,
  boxrule=0mm,
  borderline west={3.5pt}{0pt}{primary},
  sharp corners,
  breakable,
  left=15pt,
  right=0pt,
  top=0pt,
  bottom=0pt,
  toptitle=0pt,
  bottomtitle=0pt,
  colback=white,
  overlay={
    % Main accent line
    \draw[line width=1pt, accent] ([xshift=5pt]frame.north west) -- ([xshift=5pt]frame.south west);
    
    % Top decorative elements
    \fill[accent] ([xshift=0pt,yshift=-3pt]frame.north west) circle (2pt);
    \draw[line width=0.75pt, accent] ([xshift=7pt,yshift=-5pt]frame.north west) -- ([xshift=15pt,yshift=-5pt]frame.north west);
    
    % Bottom decorative elements
    \fill[accent] ([xshift=0pt,yshift=3pt]frame.south west) circle (2pt);
    \draw[line width=0.75pt, accent] ([xshift=7pt,yshift=5pt]frame.south west) -- ([xshift=15pt,yshift=5pt]frame.south west);
  }
}

\let\origfigure\figure
\let\endorigfigure\endfigure

\renewenvironment{figure}[1][htbp]{%
  \origfigure[#1]%
  \begin{figureframe}%
}{%
  \end{figureframe}%
  \endorigfigure%
}

% Styled table environment with decorative side line - same style
\newtcolorbox{tableframe}{
  enhanced,
  frame hidden,
  boxrule=0mm,
  borderline west={3.5pt}{0pt}{primary},
  sharp corners,
  breakable,
  left=15pt,
  right=0pt,
  top=0pt,
  bottom=0pt,
  toptitle=0pt,
  bottomtitle=0pt,
  colback=white,
  overlay={
    % Main accent line
    \draw[line width=1pt, accent] ([xshift=5pt]frame.north west) -- ([xshift=5pt]frame.south west);
    
    % Top decorative elements
    \fill[accent] ([xshift=0pt,yshift=-3pt]frame.north west) circle (2pt);
    \draw[line width=0.75pt, accent] ([xshift=7pt,yshift=-5pt]frame.north west) -- ([xshift=15pt,yshift=-5pt]frame.north west);
    
    % Bottom decorative elements
    \fill[accent] ([xshift=0pt,yshift=3pt]frame.south west) circle (2pt);
    \draw[line width=0.75pt, accent] ([xshift=7pt,yshift=5pt]frame.south west) -- ([xshift=15pt,yshift=5pt]frame.south west);
  }
}

\let\origtable\table
\let\endorigtable\endtable
\renewenvironment{table}[1][htbp]{%
  \origtable[#1]%
  \begin{tableframe}%
}{%
  \end{tableframe}%
  \endorigtable%
}

\begin{document}

% Cover Page
\includepdf[pages=1]{cover page.pdf}

% Blank Page after Cover - ensure it's actually blank
\newpage
\thispagestyle{empty}
\null
\newpage

% Start Arabic page numbering
\pagenumbering{arabic}

% Summary and Abstract
% Summary and Abstract on the same page
\thispagestyle{empty} % Remove page style for this page
\vspace{2.5cm}

% Create a more elegant title for Summary
\begin{center}
{\color{primary}\Large\textbf{\MakeUppercase{Summary}}}
\end{center}
\vspace{0.1cm}
\begin{center}
\rule{0.6\textwidth}{1pt}
\end{center}
\vspace{0.5cm}

\noindent \firstparagraph{T}his work is part of our Final Year Project at the Higher Institute of Computer Science and Mathematics of Monastir for 2024-2025. Conducted at \textbf{\textcolor{primary}{<<KZ IT Services>>}}, we developed \textbf{\textcolor{primary}{<<KORPOR>>}}, a real estate investment platform with a mobile app and web back-office. Using MySQL, Express-Node.js, React, and Vite, the platform offers fractional property ownership with AI for valuations and recommendations. Blockchain technology secures transactions while SCRUM methodology guided our development process.

% Keywords in a more professional design
\vspace{0.3cm}
\begin{tcolorbox}[
    colback=background,
    colframe=primary,
    arc=1mm,
    boxrule=0.5pt,
    left=8pt,
    right=8pt,
    top=4pt,
    bottom=4pt,
    width=\textwidth
]
\textbf{Keywords:} Blockchain Technology, AI, MySQL, Express-Node.js, React, Vite, Real Estate Investment.
\end{tcolorbox}

\vspace{3cm} % Space between sections

% Create a more elegant title for Abstract
\begin{center}
{\color{primary}\Large\textbf{\MakeUppercase{Abstract}}}
\end{center}
\vspace{0.1cm}
\begin{center}
\rule{0.6\textwidth}{1pt}
\end{center}
\vspace{0.5cm}

\noindent \firstparagraph{C}e projet s'inscrit dans le cadre de notre projet de fin d'études à l'Institut Supérieur d'Informatique et de Mathématiques de Monastir pour 2024-2025. Réalisé chez \textbf{\textcolor{primary}{<<KZ IT Services>>}}, nous avons développé \textbf{\textcolor{primary}{<<KORPOR>>}}, une plateforme d'investissement immobilier avec application mobile et back-office web. Utilisant MySQL, Express-Node.js, React et Vite, la plateforme permet la propriété fractionnée avec IA pour évaluations et recommandations. La blockchain sécurise les transactions tandis que la méthodologie SCRUM a guidé notre processus de développement.

% Keywords in a more professional design
\vspace{0.3cm}
\begin{tcolorbox}[
    colback=background,
    colframe=primary,
    arc=1mm,
    boxrule=0.5pt,
    left=8pt,
    right=8pt,
    top=4pt,
    bottom=4pt,
    width=\textwidth
]
\textbf{Keywords:} Blockchain Technology, AI, MySQL, Express-Node.js, React, Vite, Investissement Immobilier.
\end{tcolorbox}

% Ensure Summary and Abstract stay on the same page
\nopagebreak 
\cleardoublepage

% Dedication
\chapter*{\centering Dedication}
\addcontentsline{toc}{chapter}{Dedication}

\begin{center}
{\itshape\large 
To the memory of my beloved father, whose guidance and wisdom continue to light my path. \\
Though no longer with us, your presence remains in every achievement of my life.

\vspace{0.8cm}

To my loving mother, whose strength and endless support shaped who I am today.

\vspace{0.8cm}

To my sister and brother, whose companionship and encouragement \\
have been constant sources of joy and motivation.

\vspace{0.8cm}

To my little Aryouma, whose innocence and love bring happiness to our family every day.

\vspace{0.8cm}

To Mme Nadia, my professors and mentors, who have guided me \\
with knowledge and patience throughout my academic journey.

\vspace{0.8cm}

To my friends, whose encouragement made this journey worthwhile.

\vspace{0.8cm}

\textbf{This work is dedicated to all of you, but especially to you, Father.}
}
\end{center}

\vspace{0.5cm}

% \begin{center}
% \includegraphics[width=0.2\textwidth]{images/qr-code-dedication.png}
% \end{center}

% \vspace{1cm}

\begin{flushright}
    \begin{minipage}{0.4\textwidth}
        \centering
        \includegraphics[width=1\textwidth]{images/Ahmed_jaziri_signature.png}\\
    \end{minipage}
\end{flushright}

\thispagestyle{empty} 
\cleardoublepage

% Acknowledgement
\chapter*{\centering Acknowledgement}
\addcontentsline{toc}{chapter}{Acknowledgement}
% Add your Acknowledgement content here later 
\cleardoublepage

% Table of Contents with enhanced styling
\thispagestyle{empty}

% Decorative frame for TOC page
\begin{tikzpicture}[remember picture, overlay]
  % Draw a subtle decorative line at top of page
  \draw[color=accent, line width=0.5pt] 
    ([yshift=-1.5cm]current page.north west) -- 
    ([yshift=-1.5cm]current page.north east);
  % Draw a matching line at bottom of page  
  \draw[color=accent, line width=0.5pt] 
    ([yshift=1.5cm]current page.south west) -- 
    ([yshift=1.5cm]current page.south east);
\end{tikzpicture}

\begin{center}
{\color{primary}\LARGE\textbf{\MakeUppercase{Table of Contents}}}

% Decorative element below title
\begin{tikzpicture}
  \draw[color=primary, line width=1pt] (0,0) -- (6,0);
  \draw[color=accent, line width=0.5pt] (1,0.15) -- (5,-0.15);
\end{tikzpicture}
\end{center}

% Hide the default "Contents" heading and remove any default horizontal line
\renewcommand{\contentsname}{}
\renewcommand{\cftaftertoctitle}{}

% Return to simple TOC generation
\tableofcontents

\cleardoublepage

% General Introduction
\thispagestyle{empty}

\vspace*{\fill}

\begin{center}
\begin{minipage}{1\textwidth}
    \begin{center}
        \large\itshape ``Real estate cannot be lost or stolen, nor can it be carried away. Purchased with common sense, paid for in full, and managed with reasonable care, it is about the safest investment in the world.''\cite{RooseveltRealEstateQuote}
        \vspace{1cm}
        
        \normalfont\textcolor{primary}{— Franklin D. Roosevelt}\\
        \small\textcolor{secondary}{32nd President of the United States}
    \end{center}
\end{minipage}
\end{center}

\vspace*{\fill}

\newpage
\thispagestyle{empty}

\chapter*{\centering General Introduction}
\addcontentsline{toc}{chapter}{General Introduction}
\vspace{-1cm} % Reduce space after title

% \vspace{0.1cm}
% \begin{center}
% % \rule{0.6\textwidth}{1pt}
% \end{center}
% \vspace{0.3cm}

% \vspace{1cm}

\begingroup % Begin group to localize changes
\setlength{\parindent}{0pt} % Remove paragraph indentation
\setlength{\parskip}{0.15em} % Reduce space between paragraphs even more
\footnotesize % Smaller font size to fit on one page

\firstparagraph{I}n today's rapidly evolving financial landscape, traditional investment methods are often burdened by opaque processes, cumbersome bureaucracy, and significant entry barriers. Investors have long struggled with outdated systems that impede transparency, elevate risks, and complicate access to promising opportunities. Such challenges not only limit diversification but also expose users to uncertainties that modern technology can easily overcome.

\noindent \textbf{\textcolor{primary}{Korpor}} was conceived to transform this paradigm by delivering a fully integrated, AI and blockchain-powered mobile investment platform. By harnessing advanced data analytics, machine learning, and cutting-edge blockchain technology, Korpor streamlines every facet of the investment process. The application offers a seamless user onboarding experience, intuitive project listings enriched with AI-driven recommendations, and a secure, automated investment flow that simplifies transactions while ensuring that every operation is recorded immutably on the blockchain. Investors can manage their portfolios effortlessly through a comprehensive dashboard, with real-time notifications, an interactive AI chatbot, and multi-language support delivering a personalized and globally accessible experience.

\noindent \textbf{\textcolor{primary}{Security and trust}} are at the heart of Korpor's design. By employing state-of-the-art encryption, blockchain-based transparency, and strict compliance measures, the platform safeguards sensitive financial data and guarantees that every transaction is executed within a secure and verifiable framework. Continuous monitoring, performance optimization, and the immutable nature of blockchain records further ensure that the application remains resilient, scalable, and resistant to fraud in a dynamic market environment.

\noindent Developed under a flexible Agile framework that combines iterative development with strategic project management best practices, Korpor is designed to rapidly adapt to evolving market trends and user needs. This methodical approach allows for regular feedback, swift enhancements, and the seamless integration of innovative features throughout the development lifecycle.


% Our document is structured as follows:

% \vspace{-0.4cm} % Reduce space before list
\begin{itemize}[leftmargin=1.5em, itemsep=0pt, parsep=0pt, topsep=0.05cm]
\item The first chapter, \textbf{\textcolor{primary}{Project Context}}, delves into the industry challenges and the vision that inspired Korpor's creation.
\item The second chapter, \textbf{\textcolor{primary}{Analysis and Specification of Requirements}}, outlines the comprehensive requirements gathering, needs analysis, architectural design, and the selection of cutting-edge tools and technologies.
\item Subsequent chapters document the progressive implementation of core features—from AI-enhanced project recommendations and blockchain-secured transactions to comprehensive portfolio management—each developed through clearly defined sprints encompassing analysis, design, and deployment phases.
\end{itemize}

\vspace{0.05cm}
\noindent Through this structured approach, we demonstrate how Korpor leverages modern technology to reimagine investment management, offering a secure, transparent, and dynamic solution that is set to redefine digital financial engagement.

\endgroup % End group to restore normal settings
\enlargethispage{4cm} % Allow this page to be longer
\cleardoublepage

% Chapter 1: Project Context
\chapter{Project Context}

\section{Introduction}

The aim of this chapter is to present the general framework of the Korpor project, a solution dedicated to real estate investment. In this section, we'll discuss successively:

\begin{itemize}
    \item The presentation of the host organization.
    \item The context and challenges of the real estate sector.
    \item Analysis of existing solutions and identification of their limitations.
    \item Definition of functional and non-functional needs of the mobile application and web back office.
\end{itemize}

\section{Project Context}

This work is part of the end-of-study project for the national diploma of Applied Bachelor's degree in Computer Science from the Higher Institute of Computer Science and Mathematics of Monastir (ISIMM) for the year 2024/2025. I had the opportunity to do my end-of-study internship at the company
``KZ IT Services'', under the supervision of Mr. Khalil Zouari.

\section{Hosting Company}

The purpose of this section is to present the company within which I developed my project.

\begin{figure}[htbp]
    \centering
    \includegraphics[width=0.2\textwidth]{images/company-logo.png}
    \caption{Hosting Company ``KZ IT Services''}
    \label{fig:hosting-company}
\end{figure}

\subsection{Hosting Company}

\textbf{\textcolor{primary}{KZ IT Service}} is a dynamic software company dedicated to delivering innovative IT solutions tailored to modern business needs. We specialize in designing and developing robust, scalable applications that drive efficiency and digital transformation. Our experienced team leverages cutting-edge technology to create customized software that exceeds client expectations. With a strong commitment to quality and continuous improvement, we build lasting partnerships based on trust and excellence. At ``KZ IT Service'', innovation is at the core of everything we do, empowering our clients to achieve sustainable growth and success.

\section{Preliminary Study}

This preliminary study provides a review of some existing investment and asset management platforms. Further, the next section identifies some key concepts that will lead to further understanding of the domain in question.

\subsection{Existing Solutions Study}

For understanding the present scenario and to clearly demarcate our goals, some renowned investment platform analyses offering similar features, including ``Aseel'' and ``Stake'', are performed \cite{G2CompetitiveAnalysis2024, AsanaCompetitiveAnalysis2024}.

\subsection{Available solutions and analysis}

\subsubsection{The Aseel Platform}

\textbf{\textcolor{primary}{Aseel}} is a portal through which users can invest in different real estate projects with ease. The interface allows the clients to surf various investment opportunities, view the details of the properties, and then make an informed decision. Aseel introduces transparency in the investment process by offering financial data, updates regarding projects, and returns that are estimated. This platform comes with an easy-to-use dashboard through which one tracks their investments and manages their assets without any hassle.

\begin{figure}[htbp]
    \centering
    \includegraphics[width=0.65\textwidth]{images/Interface-of-the Aseel Platform.png}
    \caption{Interface of ``The Aseel Platform''}
    \label{fig:aseel-platform}
\end{figure}

\begin{center}
    \vspace{0.5em}
    \begin{tikzpicture}
        \draw[primary, line width=1pt] (0,0) -- (0.3\textwidth,0);
        \draw[accent, line width=0.7pt] (0.1\textwidth,-0.07) -- (0.4\textwidth,0.07);
        \fill[primary] (0,0) circle (2pt);
        \fill[accent] (0.3\textwidth,0) circle (2pt);
    \end{tikzpicture}
    \vspace{0.5em}
\end{center}

\subsubsection{The Stake Platform}

\textbf{\textcolor{primary}{Stake}} is an online investment platform that deals with real estate crowdfunding. It provides the opportunity to invest in fractions of property ownership, hence diversifying a portfolio without huge capital. On Stake, there are AI-powered recommendations based on user preferences, seamless payment integration, and a secure environment for investment. Besides, liquidity is guaranteed by enabling exit options for investors who may want to sell their shares in ongoing projects.

\newpage
\thispagestyle{empty}
\newpage

\begin{figure}[htbp]
    \centering
    \includegraphics[width=0.65\textwidth]{images/Interface-of-the Stake Platform.png}
    \caption{Interface of ``The Stake Platform''}
    \label{fig:stake-platform}
\end{figure}

\subsection{Comparative and Critical Analysis}

We can summarize all that comes from our analysis based on a number of criteria used for the evaluation of these applications \cite{LyssnaUXAnalysis2024, StrategicManagementInsightCA2024}.

\begin{itemize}
    \item \textbf{Speed (C1)}: The platform should obtain value for the user as fast as possible and effectively, anticipating their proliferating expectations.
    \item \textbf{Costs (C2)}: With minimum software development costs, it is important to keep the pricing predictable and acceptable.
    \item \textbf{Quality (C3)}: Since the market expects quality, any kind of error might affect brand reputation. Improvement of the platform should be regular.
    \item \textbf{Reliability (C4)}: Since modern-day investment platforms need to make sure of minimum downtime and maximum availability of services, this factor is critical.
    \item \textbf{Security (C5)}: Such an investment platform enforces access rights, roles, and contribution rights through a powerful security system.
    \item \textbf{Performance (C6)}: Crucial features include AI-powered recommendations going through seamlessly, easy transaction tracking, and investment monitoring.
    \item \textbf{Stability (C7)}: The platform should have a proven track record, regular updates, and a large user base to ensure its longevity.
    \item \textbf{Resilience (C8)}: In order to prevent data loss and guarantee a smooth experience for investors, it must be able to restore lost functionalities should issues occur.
    \item \textbf{User Experience (C9)}: The interface should be intuitive and user-friendly, hence allowing investors to move with ease through it, thus making wiser decisions.
\end{itemize}

\begin{table}[htbp]
    \centering
    \caption{Evaluation Table}
    \label{tab:evaluation}
    \renewcommand{\arraystretch}{1.3}
    \begin{tabular}{|>{\columncolor{background}}c|*{9}{>{\centering\arraybackslash}p{0.7cm}|}}
        \hline
        \rowcolor{primary!15}
        \textcolor{primary}{\textbf{Solution}} & 
        \textcolor{primary}{\textbf{C1}} & 
        \textcolor{primary}{\textbf{C2}} & 
        \textcolor{primary}{\textbf{C3}} & 
        \textcolor{primary}{\textbf{C4}} & 
        \textcolor{primary}{\textbf{C5}} & 
        \textcolor{primary}{\textbf{C6}} & 
        \textcolor{primary}{\textbf{C7}} & 
        \textcolor{primary}{\textbf{C8}} & 
        \textcolor{primary}{\textbf{C9}} \\
        \hline
        \textbf{Stake} & \textcolor{primary}{\checkmark} & \textcolor{primary}{\checkmark} & \textcolor{primary}{\checkmark} & \textcolor{primary}{\checkmark} & \textcolor{primary}{\checkmark} & $\times$ & \textcolor{primary}{\checkmark} & \textcolor{primary}{\checkmark} & \textcolor{primary}{\checkmark} \\
        \hline
        \rowcolor{background!50}
        \textbf{Aseel} & \textcolor{primary}{\checkmark} & \textcolor{primary}{\checkmark} & \textcolor{primary}{\checkmark} & $\times$ & \textcolor{primary}{\checkmark} & $\times$ & \textcolor{primary}{\checkmark} & $\times$ & \textcolor{primary}{\checkmark} \\
        \hline
    \end{tabular}
\end{table}

\subsection{Proposed Solution}

Having studied the already working platforms for investments, we found strengths and weaknesses that could define what was required from the project. Our proposed solution will look at:

\begin{itemize}
    \item Developing an efficient mobile application for investment management.
    \item Increasing the level of users' engagement with recommendations using the power of AI \cite{MobileRealityAIRealEstate2024, HouseCanaryAIInvestors2024}.
    \item Ensuring responsive and user-friendly interaction with the interface.
    \item Gaining the trust of investors by ensuring transparency and security in the investing platform.
    \item Enhancing the security of data and following all the financial regulations.
\end{itemize}

The \textbf{\textcolor{primary}{Korpor}} platform will be offering the following features:

\begin{itemize}
    \item A directory of investment opportunities with deep financial insights into those opportunities.
    \item AI-driven recommendations of investments as per users' preferences \cite{LinkedInAIFinTech2024}.
    \item Smooth funding and payout mechanisms.
    \item Real-time portfolio performance tracking on a single screen/dashboard.
    \item Forum for interactive discussions on strategy and market trends among its users.
    \item Referral and Rewards System: An engaging system for rewarding users via referral.
\end{itemize}

\section{Development methodology}

The completion of the project on its delivery date is the main problem of every software development team. One of the most common problems encountered in the production of software is insufficient technical specifications, poor time management in the face of the use of emerging technology, and sudden changes in needs. In order to avoid these critical issues, we follow an agile methodology for project management.

\subsection{SCRUM}

\textbf{\textcolor{primary}{Scrum}} is an agile development approach that is used to create software using incremental and iterative methods. Scrum is a quick, flexible, and efficient agile methodology that is intended to provide value to the client at every stage of the project's development \cite{ScrumGuide2020}. Scrum is founded on empiricism and lean thinking, employing an iterative, incremental approach guided by the three pillars of transparency, inspection, and adaptation \cite{AtlassianScrumPillars, ScrumGuide2020}. Scrum's main goal is to meet customer needs by fostering an atmosphere of open communication, group accountability, and constant improvement, underpinned by the Scrum values of Commitment, Focus, Openness, Respect, and Courage \cite{ScrumGuide2020}. The development process begins with a broad concept of what must be constructed, developing a list of features that the product owner desires, and arranging them according to priority (product backlog).

\subsection{Agile Scrum roles and responsibilities}

\subsubsection{The Product Owner}

Understands the customer and business requirements, then creates and manages the product backlog based on those requirements.

\textbf{Responsibilities:}
\begin{itemize}
    \item Managing the scrum backlog
    \item Release management
    \item Stakeholder management
\end{itemize}

\subsubsection{Developers}

In Scrum, the term developer or team member refers to anyone who plays a role in the development and support of the product and can include researchers, architects, designers, programmers, etc.

\textbf{Responsibilities:}
\begin{itemize}
    \item Delivering the work through the sprint
    \item To ensure transparency during the sprint, they meet daily at the daily scrum
\end{itemize}

\subsubsection{Scrum Master}

The role responsible for gluing everything together and ensuring that scrum is being done well. In practical terms, that means they help the product owner define value, the development team deliver the value, and the scrum team get better.

The Scrum Master focuses on:
\begin{itemize}
    \item Transparency
    \item Empiricism
    \item Self-organization
    \item The Scrum events
\end{itemize}

\subsection{The Scrum Events}

The Scrum events are key elements of the Scrum Framework. They provide regular opportunities for enacting the Scrum pillars of Inspection, Adaptation and Transparency \cite{ScrumGuide2020}. In addition, they help teams keep aligned with the Sprint and Product Goals, improve Developer productivity, and remove impediments and reduce the need to schedule too many additional meetings.

\begin{itemize}
    \item \textbf{Sprint}: All work in Scrum is done in a series of short projects called Sprints. This enables rapid feedback loops.
    
    \item \textbf{Sprint Planning}: The Sprint starts with a planning session in which the Developers plan the work they intend to do in the Sprint. This plan creates a shared understanding and alignment among the team.
    
    \item \textbf{Daily Scrum}: The Developers meet daily to inspect their progress toward the Sprint Goal, discuss any challenges they've run into, and tweak their plan for the next day as needed.
    
    \item \textbf{Sprint Review}: At the end of the Sprint, the Scrum Team meets with stakeholders to show what they have accomplished and get feedback.
    
    \item \textbf{Sprint Retrospective}: Finally, the Scrum Team gets together to discuss how the Sprint went and if there are things they could do differently and improve in the next Sprint.
\end{itemize}

\section{Conclusion}

In conclusion of this chapter, it is clear that planning and methodology are essential pillars to ensure the success of the project. By fully understanding the project framework, including the host organization's expectations and the challenges ahead, the team is better prepared to meet the challenges ahead.

This chapter lays the solid foundation on which the entire project will be built, providing a valuable guide for the next steps. The next chapter will allow us to analyze and specify the needs developed in our project. 
\cleardoublepage

% Insert page break in TOC before Chapter 2
\addtocontents{toc}{\protect\newpage}

% Chapter 2: Analysis and Specification
\chapter{Analysis and Specification of Needs}

\section{Introduction}

In this chapter, we will present the analysis and specification of needs. We start by presenting the specification of the requirements, illustrating them using the diagram of the global use cases. Then we will present our project architecture and our working environment, and finally we will present our product backlog and releases planning, and we will close our chapter with a conclusion.

\section{Requirements Specification}

In this section, we will define the actors of our application and the functional and non-functional needs that our application aims to fulfill.

\subsection{Identifying Actors}

We define actors as a shorthand for the roles played by entities outside the system that interact directly with them. In our system, we identify four types of actors:

\begin{itemize}
    \item \textbf{\textcolor{primary}{Super Admin}}: Responsible for the global configuration of the platform, they have extended privileges to manage administrators, oversee security, and ensure compliance. They can also configure advanced features and control all system resources.
    
    \item \textbf{\textcolor{primary}{Admin}}: In charge of the day-to-day management of the platform, they can add, modify, or delete listings, supervise agency and user profiles, and ensure smooth operations. They are also responsible for monitoring and assisting other actors.
    
    \item \textbf{\textcolor{primary}{Real Estate Agent}}: Dedicated to creating and updating real estate listings, they manage property information, handle investor requests, and finalize transactions related to sales or rentals. They can also coordinate property visits and propose tailored offers.
    
    \item \textbf{\textcolor{primary}{Investor}}: A user who wishes to browse and finance real estate projects. They have access to all available offers, can make investments in a few simple steps, and monitor the evolution of their portfolio. They also benefit from personalized insights to optimize their investments.
    
    \item \textbf{\textcolor{primary}{System}}: The entity that automatically manages all basic functionalities, such as authentication, notification generation, transaction validation, and adherence to security protocols. It ensures the coherence and reliability of the application at all times.
\end{itemize} 

\subsection{Functional Requirements}

After several meetings with our client, the various functional requirements of our application are illustrated as follows:

\subsubsection{For the Super Admin (Korpor)}
\begin{itemize}
    \item \textbf{Authenticate}: The super admin enters their credentials to access the advanced management console.
    \item \textbf{Log Out}: After viewing or updating global settings, they can securely log out.
    \item \textbf{Manage Admin Accounts}: Create, enable/disable, or modify admin profiles associated with different real estate companies.
    \item \textbf{Monitor Security \& Compliance}: Oversee transactions, data integrity, and regulatory adherence using specialized reporting and audit tools.
    \item \textbf{Configure Platform Features}: Define key parameters (payment methods, AI/blockchain integrations, etc.) and roll out feature updates.
    \item \textbf{View Global Reports}: Generate and analyze consolidated metrics (financials, user activity, transactions) for overall performance insights.
    \item \textbf{Moderate Content}: Review and remove any inappropriate or erroneous property listings or user-generated data.
\end{itemize} 

\subsubsection{For the Admin (Real Estate Company)}
\begin{itemize}
    \item \textbf{Authenticate}: The admin logs in with valid credentials to manage daily operations.
    \item \textbf{Log Out}: They can end their session to maintain account security.
    \item \textbf{Manage Real Estate Listings}: Add, update, or delete property listings visible to investors.
    \item \textbf{Oversee Real Estate Agents}: Create and manage agent accounts, assign properties, and monitor performance and commissions.
    \item \textbf{Track Transactions \& Commissions}: Review incoming payments, calculate commissions owed to agents, and track the history of completed deals.
    \item \textbf{Address Investor Inquiries}: Respond to questions or concerns from investors, ensuring a smooth user experience.
    \item \textbf{Access Agency Dashboard}: View comprehensive statistics on properties, sales, rentals, and market trends.
\end{itemize}

\subsubsection{For the Real Estate Agent}
\begin{itemize}
    \item \textbf{Authenticate}: The agent logs in to manage assigned properties and interact with potential investors.
    \item \textbf{Log Out}: Securely exit the account after completing tasks.
    \item \textbf{Manage Assigned Properties}: Create new listings, update property details, set prices, and upload images.
    \item \textbf{Handle Investment Requests}: Review purchase or rental offers, negotiate terms, and initiate contract finalization.
    \item \textbf{Contribute to AI Estimates}: Provide or refine data to improve AI-driven pricing and market analysis.
    \item \textbf{Maintain Client Relationships}: Communicate with investors, schedule property visits, and follow up on inquiries.
    \item \textbf{View Commissions}: Track earnings based on successful sales or rentals.
\end{itemize}

\subsubsection{For the Investor (Mobile App User)}
\begin{itemize}
    \item \textbf{Create an account \& authenticate}: Register to gain access to the platform's core features.
    \item \textbf{Log Out}: End the session to protect personal and financial data.
    \item \textbf{Browse Listings \& Invest}: Explore available properties, filter according to preferences, and commit to an investment in a few steps.
    \item \textbf{Track Portfolio}: Monitor owned assets, property status, and receive real-time updates on performance.
    \item \textbf{Make Payments}: Use integrated payment methods (credit cards, digital wallets, etc.) to complete transactions.
    \item \textbf{Access AI Recommendations}: View data-driven insights and return-on-investment estimates generated by the system.
    \item \textbf{Manage Withdrawals \& Earnings}: Withdraw profits, monitor rental income, or exit investments under the right conditions.
\end{itemize}

\subsubsection{For the System}
\begin{itemize}
    \item \textbf{Automate Authentication}: Validate credentials, manage sessions, and maintain user roles and permissions.
    \item \textbf{Generate Notifications}: Send real-time alerts (e.g., new listings, completed transactions, commission updates) to relevant users.
    \item \textbf{Ensure Compliance \& Security}: Leverage blockchain for data integrity, verify payments, and detect anomalies or fraudulent activities.
    \item \textbf{Coordinate AI Insights}: Aggregate and analyze real estate data to produce market predictions and price recommendations.
    \item \textbf{Maintain Transaction Consistency}: Update dashboards, user balances, and property statuses automatically upon each operation.
    \item \textbf{Optimize Performance}: Monitor server load, scale resources, and ensure a smooth, responsive application experience.
\end{itemize} 

\subsection{Non-functional Requirements}

In order to ensure the proper functioning of the decision-making system and to avoid any kind of anomaly, the implemented solution must meet a set of non-functional needs such as:

\begin{itemize}
    \item \textbf{Maintainability}: The system must be designed for simplicity so that tasks, updates, and bug fixes can be executed with minimal complexity.
    
    \item \textbf{Evolution}: Platform administration must remain attentive to user needs and feedback, continuously enhancing the services offered while preserving the application's utility and efficiency.
    
    \item \textbf{Security}: Robust security measures are essential. The platform must enforce strong authentication protocols, access privileges, and comprehensive data encryption (both at rest and in transit). The integration of blockchain technology further ensures the immutability and integrity of sensitive information.
    
    \item \textbf{Efficiency}: The application must be effective in all circumstances, delivering prompt and reliable functionality regardless of external conditions.
    
    \item \textbf{Performance}: The system must operate optimally across diverse environments. It should consistently provide a responsive and reliable experience, even under high transaction volumes or varying network conditions.
\end{itemize} 

\section{Requirements Analysis}

In this section, we'll outline the various features that our app should offer, using a general use case diagram.

\subsection{General use case diagram}

Below, we present the various actors of the application and the actions they are authorized to perform.
The overall diagram is illustrated in the following figure:

\begin{figure}[htbp]
    \centering
    % Replace with actual image file once available
    \includegraphics[width=0.85\textwidth]{images/diagram de case d utilisation general.png}
    \caption{General use case diagram}
    \label{fig:use-case-diagram}
\end{figure}
\cleardoublepage

% Insert page break in TOC before Chapter 3
% \addtocontents{toc}{\protect\newpage}

% Chapter 3: Design and Implementation
\chapter{Foundation}

\section*{Introduction}
This chapter details the implementation phase of our project, which follows an agile methodology with four sprints. Each sprint focuses on delivering specific features and functionality according to the project backlog. The implementation utilizes a modern tech stack consisting of React \cite{ReactWebsite} with Vite \cite{ViteJSWebsite} for the frontend, Node.js \cite{NodeJSWebsite} with Express.js \cite{ExpressJSWebsite} for the backend, MySQL \cite{MySQLWebsite} for database management, and Tailwind CSS \cite{TailwindWebsite} for styling.

Our first sprint focused on establishing essential foundational components of the system, with the following key deliverables:

\subsection{Web Backoffice Authentication}
\begin{itemize}
    \item Admin dashboard login and authentication system
    \item Role-based access control for backoffice users (Super Admin, Admin, Agent)
    \item User management interface for creating and managing user accounts
    \item Permission management system for different user roles
    \item Security logs and audit trails for backoffice activities
    \item Session management and secure token handling
\end{itemize}

\section{Sprint 1: Authentication and User Management }
\subsection{Overview}
The first sprint focuses on establishing the core authentication system and user management functionality. This foundation is critical for all subsequent features as it defines user roles and access controls.


\subsection{Authentication and User Management System}
The Authentication and User Management system forms the backbone of Korpor's security and user interaction model. It encompasses processes for user registration with role assignment, secure login with session and token management, password recovery mechanisms, and session termination. Figure \ref{fig:auth-usermgmt-usecase} illustrates the global use cases for these core functionalities.

\begin{figure}[htbp]
    \centering
    % Replace with actual image file path
    \includegraphics[width=0.9\textwidth]{images/auth-usermgmt-usecase.png} 
    \caption{Global Use Case Diagram for Authentication and User Management}
    \label{fig:auth-usermgmt-usecase}
\end{figure}
\newpage

\subsubsection{Sign-up Process}
The sign-up process is illustrated in Figure \ref{fig:signup-diagram} below. The diagram shows the authentication flow for new users registering in the system.

\begin{figure}[ht!]
    \centering
    \includegraphics[width=1.03\textwidth]{images/diagram_de_case_d_utilisation_signup.png}
    \caption{Authentication Sign-up Use Case Diagram}
    \label{fig:signup-diagram}
\end{figure}

Figure \ref{fig:signup-activity-diagram} illustrates the activity flow of the sign-up process. The process begins when a new user navigates to the sign-up page and presses the sign-up button. The system then presents a form for the user to fill out. Upon submission, the system validates the entered information. 

If the information is invalid, the user is prompted to correct the form. If the information is valid, the system saves the user's details in the database and sends a verification email to the user. The user must then check their email. 

Upon successful email verification, the system redirects the user to an "email verification code" page or a similar confirmation step. The Super Admin is then involved in a two-step verification process. If the Super Admin accepts the user, their login is enabled. If the Super Admin refuses the user, the user's account is deleted. If the initial email verification step fails (e.g., there's an issue with the verification code), an error message is displayed to the user.

\newpage

\begin{figure}[ht!]
    \centering
    \includegraphics[width=1\textwidth]{images/signup_activitydiag.png}
    \caption{Authentication Sign-up Activity Diagram}
    \label{fig:signup-activity-diagram}
\end{figure}


% \vspace{1cm}

\subsubsection{Login Process}
The login process allows existing users to access the system. The table below details the use case.

\begin{table}[htbp]
    \centering
    \begin{tabular}{|l|p{0.7\textwidth}|}
        \hline
        \textbf{Section} & \textbf{Details} \\
        \hline
        Use Case & User Login \\
        \hline
        Actor & User (Super Admin, Admin, Agent, Investor) \\
        \hline
        Precondition & User has an existing, verified account. User is on the login page. \\
        \hline
        Main Scenario & 
        1. User enters their credentials (e.g., email and password).
        2. User clicks the login button.
        3. System verifies the credentials.
        4. If credentials are valid, the system grants access and redirects the user to their respective dashboard. \\
        \hline
        Postcondition & User is successfully logged into the system and can access features based on their role. \\
        \hline
        Exception & 
        - Invalid credentials: System displays an error message.
        - Account locked/disabled: System displays an appropriate message.
        - System error: System displays a general error message. \\
        \hline
    \end{tabular}
    \caption{Login Process Details}
    \label{tab:login_process}
\end{table}

\newpage
\subsubsection{Manage Users Process}
The user management process enables administrators to create, view, update, and delete user accounts. The table below details the use case.

\begin{table}[htbp]
    \centering
    \begin{tabular}{|l|p{0.7\textwidth}|}
        \hline
        \textbf{Section} & \textbf{Details} \\
        \hline
        Use Case & Manage User Accounts \\
        \hline
        Actor & Super Admin \\
        \hline
        Precondition & Actor is logged into the system with appropriate administrative privileges. \\
        \hline
        Main Scenario & 
        1. Actor navigates to the user management section.
        2. To create a user: Actor fills in user details (name, email, role, etc.) and submits the form. System creates the new user account.
        3. To view users: System displays a list of users. Actor can search/filter the list.
        4. To update a user: Actor selects a user, modifies their details, and saves the changes. System updates the user account.
        5. To delete a user: Actor selects a user and confirms deletion. System deactivates or deletes the user account. \\
        \hline
        Postcondition & User account is created, updated, or deleted as per the action taken. The list of users reflects the changes. \\
        \hline
        Exception & 
        - Invalid input data: System displays validation errors.
        - Permission denied: System prevents unauthorized actions.
        - User not found (for update/delete): System displays an error message.
        - System error: System displays a general error message. \\
        \hline
    \end{tabular}
    \caption{Manage Users Process Details}
    \label{tab:manage_users_process}
\end{table}


The sign-up process includes user registration, role assignment, and account verification steps. During registration, users are categorized into one of the three user types: Super Admin, Admin, or Agent, with each type having different permissions and access levels within the system.

\newpage    

\begin{figure}[htbp]
    \centering
    \includegraphics[width=1\textwidth]{images/auth_classdiag.PNG}
    \caption{Authentication AND User Management System Class Diagram}
    \label{fig:auth-class-diagram}
\end{figure}

Figure \ref{fig:auth-class-diagram} depicts the class diagram for the authentication system. It showcases the key components and their relationships involved in user authentication and authorization.

\newpage

\begin{itemize}
    \item \textbf{AuthComponents}: Represents the UI elements for authentication, such as SignInForm, SignUpForm, ForgotPasswordForm, ResetPasswordForm, and OTPVerification. These components are used by \textbf{MobileAuthScreens}.
    \item \textbf{MobileAuthScreens}: Includes screens like LoginScreen, SignupScreen, and ForgotPasswordScreen that utilize \textbf{AuthComponents} and interact with the \textbf{AuthService}.
    \item \textbf{AuthService}: Acts as an intermediary between the frontend components/screens and the backend. It handles functions like signIn, signUp, verifyEmail, forgotPassword, resetPassword, logout, refreshToken, and error handling. It consumes \textbf{AuthRoutes}.
    \item \textbf{AuthRoutes}: Defines the API endpoints for authentication, such as /login, /register, /verify-email, /forgot-password, /reset-password, /logout, and /refresh-token. These routes map to methods in the \textbf{AuthController}.
    \item \textbf{AuthController}: Contains the core logic for authentication processes, including login, register, verifyEmail, resetPassword, changePassword, logout, refreshToken, and sendVerificationEmail. It interacts with the \textbf{User}, \textbf{Role}, and \textbf{BlacklistedToken} models and utilizes \textbf{AuthMiddleware}.
    \item \textbf{AuthMiddleware}: Provides middleware functions for authentication (authenticate) and authorization (authorize roles). It also manages blacklisted tokens (blacklistToken) and interacts with the \textbf{User} model.
    \item \textbf{User}: Represents the user entity with attributes like id, accountNo, name, surname, email, password, birthdate, resetCode, isVerified, approvalStatus, failedLoginAttempts, lockedUntil, refreshToken, and refreshTokenExpires. A User has one or more \textbf{Role}s.
    \item \textbf{Role}: Defines user roles with attributes like id, name, privileges (JSON), and description. Each User is associated with a Role (0..1 relationship shown, typically a User has at least one Role, but the diagram indicates a User can have zero or one Role, which might need clarification or represent a specific system design choice, e.g., a default role or a user awaiting role assignment).
    \item \textbf{BlacklistedToken}: Stores tokens that have been invalidated (e.g., after logout or password change) with attributes like id, token, and expiresAt. It includes methods like isBlacklisted and cleanupExpired. The \textbf{AuthController} uses this to validate tokens, and \textbf{AuthMiddleware} checks against it.
\end{itemize}

\subsection{Implementation Interfaces}
The implementation of the authentication and user management system resulted in the following key user interfaces:

\subsubsection{Sign-in Interface}
The sign-in interface provides a secure and user-friendly means for users to authenticate. Figure \ref{fig:signin-interface} shows the implementation of this interface.
\newpage
\begin{figure}[htbp]
    \centering
    % Add correct path when available
    \includegraphics[width=1\textwidth]{images/signin-interface.png}
    \caption{Sign-in Interface Implementation}
    \label{fig:signin-interface}
\end{figure}

\subsubsection{Sign-up Interface}
The sign-up interface allows new users to register for an account. It collects necessary information and begins the verification process. Figure \ref{fig:signup-interface} displays this implementation.

\begin{figure}[htbp]
    \centering
    % Add correct path when available
    \includegraphics[width=1\textwidth]{images/signup-interface.png}
    \caption{Sign-up Interface Implementation}
    \label{fig:signup-interface}
\end{figure}

\subsubsection{User Management in Super Admin Dashboard}
The Super Admin dashboard provides comprehensive user management capabilities, including the ability to view, create, edit, and deactivate user accounts. Figure \ref{fig:user-management} shows this powerful interface.
\newpage
\begin{figure}[htbp]
    \centering
    \includegraphics[width=1\textwidth]{images/user-management-dashboard.png}
    \caption{User Management Interface in Super Admin Dashboard}
    \label{fig:user-management}
\end{figure}

\subsection{Mobile Application Interfaces}
The mobile application provides a streamlined authentication experience for investors. The login and sign-up interfaces are designed for clarity and ease of use on mobile devices, as shown in Figure \ref{fig:mobile-auth-interfaces}.

\begin{figure}[htbp]
    \centering
    % Replace with actual image file path
    % \includegraphics[width=0.9\textwidth]{images/mobile-auth-screens.png} 
    \caption{Mobile Application Login and Sign-up Interfaces}
    \label{fig:mobile-auth-interfaces}
\end{figure}
\newpage % Ensures the next section starts on a new page if needed

\subsection{Testing Validation}
To ensure the reliability and functionality of the authentication and user management system, we implemented comprehensive testing using Playwright, an end-to-end testing framework \cite{PlaywrightDocs2023}.

\subsubsection{Playwright Test Results}
The authentication system underwent rigorous testing through automated test scripts that verified all key functionality, including sign-up, sign-in, password reset, and user management operations. Figure \ref{fig:playwright-tests} demonstrates the successful execution of these tests.
% \newpage
\begin{figure}[htbp]
    \centering
    % Add correct path when available
    \includegraphics[width=0.75\textwidth]{images/playwright-test-results.png}
    \caption{Playwright Test Results for Authentication System}
    \label{fig:playwright-tests}
\end{figure}


All tests passed successfully, confirming the robustness of the implemented authentication system.

\section*{Conclusion}

The Foundation sprint successfully established the secure authentication and user management infrastructure for the Korpor platform.

% Additional sections for subsequent sprints will be added later 
\cleardoublepage

% Main Matter (Page numbering already set to arabic)
% \chapter{Project Context}
% \input{chapters/project_context.tex}

% \chapter{Introduction} % Example for later
% \input{chapters/introduction.tex}

\end{document} 